% Template for Cogsci submission with R Markdown

% Stuff changed from original Markdown PLOS Template
\documentclass[10pt, letterpaper]{article}

\usepackage{cogsci}
\usepackage{pslatex}
\usepackage{float}
\usepackage{caption}

% amsmath package, useful for mathematical formulas
\usepackage{amsmath}

% amssymb package, useful for mathematical symbols
\usepackage{amssymb}

% hyperref package, useful for hyperlinks
\usepackage{hyperref}

% graphicx package, useful for including eps and pdf graphics
% include graphics with the command \includegraphics
\usepackage{graphicx}

% Sweave(-like)
\usepackage{fancyvrb}
\DefineVerbatimEnvironment{Sinput}{Verbatim}{fontshape=sl}
\DefineVerbatimEnvironment{Soutput}{Verbatim}{}
\DefineVerbatimEnvironment{Scode}{Verbatim}{fontshape=sl}
\newenvironment{Schunk}{}{}
\DefineVerbatimEnvironment{Code}{Verbatim}{}
\DefineVerbatimEnvironment{CodeInput}{Verbatim}{fontshape=sl}
\DefineVerbatimEnvironment{CodeOutput}{Verbatim}{}
\newenvironment{CodeChunk}{}{}

% cite package, to clean up citations in the main text. Do not remove.
\usepackage{apacite}

% KM added 1/4/18 to allow control of blind submission


\usepackage{color}

% Use doublespacing - comment out for single spacing
%\usepackage{setspace}
%\doublespacing


% % Text layout
% \topmargin 0.0cm
% \oddsidemargin 0.5cm
% \evensidemargin 0.5cm
% \textwidth 16cm
% \textheight 21cm

\title{How to Make a Proceedings Paper Submission}

\usepackage{float} \floatplacement{figure}{T} \usepackage{graphicx}
\usepackage{booktabs}
\usepackage{longtable}
\usepackage{array}
\usepackage{multirow}
\usepackage{wrapfig}
\usepackage{float}
\usepackage{colortbl}
\usepackage{pdflscape}
\usepackage{tabu}
\usepackage{threeparttable}
\usepackage{threeparttablex}
\usepackage[normalem]{ulem}
\usepackage{makecell}
\usepackage{xcolor}

\author{{\large \bf Morton Ann Gernsbacher (MAG@Macc.Wisc.Edu)} \\ Department of Psychology, 1202 W. Johnson Street \\ Madison, WI 53706 USA \AND {\large \bf Sharon J.~Derry (SDJ@Macc.Wisc.Edu)} \\ Department of Educational Psychology, 1025 W. Johnson Street \\ Madison, WI 53706 USA}

\newlength{\cslhangindent}
\setlength{\cslhangindent}{1.5em}
\newenvironment{CSLReferences}%
  {}%
  {\par}

\begin{document}

\maketitle

\begin{abstract}
The ``shape bias'' -- the bias to generalize new nouns by their shape
rather than other features such as color or texture -- has been argued
to facilitate early noun learning for children. However, there is
conflicting evidence about the magnitude and nature of this bias, and a
huge heterogeniety in the literature that is potentially masking
developmetnal and cultural differences. This study examines the
influence of procedural variation on the heterogeneity, examining the
effect of stimuli and procedural design in a small study that aims to
provide the basis for a large scale assessment. In two experiments, we
replicate the shape bias in 22 participants age-. Then, we test the same
age range using the same stimuli in a within-subject experiment
manipulating the functional cues provided. we find that

\textbf{Keywords:}
Add your choice of indexing terms or keywords; kindly use a semi-colon;
between each term.
\end{abstract}

\hypertarget{basis-of-generalizations}{%
\section{Basis of generalizations}\label{basis-of-generalizations}}

What does ``dog'' mean to a 2-year-old? This old question reflects the
evolving nature of our understanding of early label categorization. For
young children, a dog might initially be identified by a set of general
features that expands with experience (CLARK, 1973). For example, The
semantic features hypothesis (CLARK, 1973) suggests that categorization
begins with schemas of basic perceptual attributes, such as {[}four
legs, tail, fur{]} for dog, hence we see overextensions of the word
``dog'' to beings that share these set of features like cats for
example. Over time, deeper attributes, such as its internal composition,
behavior and interaction with the world, or sound, may also become
relevant (Carey, 1985). In contrast, the functional core hypothesis
(Nelson, 1974) posits that children extract relationships among features
to identify future category members without treating these features as
defining. In this view, perceptual attributes like {[}four legs, tail,
fur{]} act as identifiers rather than encapsulating the category's
essence. The defining features of the lexical category manifest in the
way the word is used to label other objects. Understanding how children
generalize a noun learned in the presence of a few exemplars remains
critical to deciphering the underlying processes of lexical concept
formation. Overextensions, which is \#FIXME, are common in early word
acquisition (Rescorla, 1980 ), yet the nature of the underlying
representations and their developmental trajectory remain debated. Both
frameworks, whether a perceptual or a conceptual one, agree that shared
perceptual features facilitate identification and labeling, forming the
foundation for the well-studied shape bias Larissa K. Samuelson \& Smith
(2000).

The shape bias, which is the tendency to generalize objects names by
their shape, rather than other properties, is a word learning constraint
that is argued to facilitate early noun acquisition, to be an important
route to vocabulary growth, and found to be weaker in children with
language delay (Susan S. Jones, 2003; JONES \& SMITH, 2005; Smith,
Jones, Landau, Gershkoff-Stowe, \& Samuelson, 2002; Tek, Jaffery, Fein,
\& Naigles, 2008 ; Tek \& Naigles, 2017). Nevertheless, the shape bias
observed in word learning experiments is highly variable across ages,
cultures, languages, and experimental conditions, leading to conflicting
outcomes and difficulties integrating the state of evidence to assess
its commensurability and form a coherent understanding of the phenomenon
of label categorization (Kucker et al., 2019).

A recent meta-analytic effect size of 0.8, derived from over 300
standardized effects across 40 studies (Abdelrahim \& Frank, 2024),
confirms the robustness of the shape bias. However, substantial
heterogeneity in the data (with over 90\% of variance unexplained by age
or language) suggests that cross-cultural, linguistic, and developmental
differences remain masked.

\hypertarget{sources-of-heterogeneity}{%
\section{Sources of Heterogeneity}\label{sources-of-heterogeneity}}

The unexplained variability seen across studies and experiments are
thought to be due to procedural variation i.e.~task format and stimuli.

\hypertarget{task-format}{%
\subsection{Task format}\label{task-format}}

Generalization in word learning is often assessed using the word
extension task. Here, children are taught a novel label for a novel
object and tested on their ability to extend it to other objects that
share features like shape, or material, etc, with the exemplar object.
Word extensions are often measured via Forced-choice tasks, which
require restrictive generalizations, yes/no endorsement tasks, allowing
broader acceptance of category membership which allows for different
levels of similarity and difference (Landau et al., 1988), or
Open-choice tasks, enabling children to reject all options, assumed to
indicate an understanding of category membership that goes beyond shared
perceptual features. Forced-choice tasks may yield different results
compared to yes/no endorsement tasks, and allowing children to select
``none of those'' reduces shape bias, especially with complex objects
(Cimpian et al 2005). The choice of the task format is often guided by
the theoretical framework of the researchers, leading to what seems like
a circular stream of events in which theory informs task selection, and
task selection confirms theory.

\hypertarget{stimuli}{%
\subsection{Stimuli}\label{stimuli}}

A significant sources of variation in the word extension findings comes
from differences in stimuli. Just like task format, these methodological
decisions are often driven by theoretical frameworks as well. For
instance, studies focusing on low-level attentional biases typically
emphasize contrasts between shape and other perceptual features, such as
color or material, using stimuli designed to highlight these attributes.
On the other hand, studies investigating conceptual understanding often
include cues related to animacy, such as eyes, shoes, or other salient
features, and use test objects that share multiple dimensions with the
exemplar instead of only one to explore broader conceptual frameworks
(Susan S. Jones \& Smith, 1998; Yoshida \& Smith, 2003b). When
functionality is emphasized, stimuli are often paired with
demonstrations of an affordance, stories, or narratives to contrast
shape with function. Children aged 2 to 5 years are frequently found to
prioritize shape, even when provided with functional information
(Centner, 2003; Graham, Williams, \& Huber, 1999; Landau \& Jones, 1998;
Merriman, Scott, \& Marazita, 1993) {[}Gentner \& Rattermann, 1991;
Woodward \& Markman, 1998; etc.). However, conflicting evidence shows
children sometimes prioritize function or other cues (Kemler Nelson,
1995; Gelman \& Medin, 1993; etc.), with variation linked to factors
like whether test objects were handled or how stimuli were designed
(e.g., functional bases vs.~appended parts). In addition, some studies
use pictures or drawings, while others use physical objects (cite).

Lastly, most studies employ between-subjects designs, which do not
control for individual differences, further amplifying heterogeneity.
These procedural and stimuli variations reflect broader theoretical
questions about the origins of the shape bias (Smith \& Medin, 1981). Is
it a Low-level attentional mechanisms driven by attentional processes
that guide children to perceptual features associated with category
labels? Or a Top-down conceptual processes in which the perceptual
features act as identifiers rather than defining properties? Where
should the line be drawn between perceptual feature identification and
the core representation of conceptual labels? Are these separate
processes, or do they exist on a continuum that develops as children
acquire more information? How do attention to perceptual attributes and
conceptual understanding interact during development (Madole \& Oakes,
1999)? These foundational questions influence procedural decisions and
should be kept in mind when investigating label categorization and
concept formation.

\hypertarget{theorectical-implications}{%
\section{Theorectical implications}\label{theorectical-implications}}

The investigation of the shape bias, and label categorization more
broadly, has unfolded around two major debates: a cross-cultural debate
and a representational debate.

\hypertarget{cross-linguistic-debate}{%
\subsection{Cross-linguistic debate}\label{cross-linguistic-debate}}

The word extension task literature highlights significant
cross-linguistic differences in the prevalence of the shape bias, which
are considered theoretically important. For example, speakers of East
Asian languages like Mandarin and Japanese demonstrate less reliance on
shape when extending nouns compared to English speakers in the United
States (Gathercole \& Min, 1997; Imai \& Gentner, 1997; Larissa K.
Samuelson \& Smith, 1999; Smith, Colunga, \& Yoshida, 2003; Nancy N.
Soja, Carey, \& Spelke, 1991; Subrahmanyam \& Chen, 2006; Yoshida \&
Smith, 2003b).

Two key hypotheses attempt to explain these differences: 1- Syntactic
Structure Hypothesis: Differences in linguistic structure, such as
count-mass syntax in English versus classifier systems in East Asian
languages, influence the prevalence of the shape bias (Imai \& Gentner,
1997; Larissa K. Samuelson, Horst, Schutte, \& Dobbertin, 2008; Nancy N.
Soja et al., 1991; Nancy N. Soja, Carey, \& Spelke, 1992).

2- Statistical Regularities Hypothesis: Variations in lexical and
environmental statistical regularities tunes attention toward features
like shape. This hypothesis emphasizes the role of existing vocabulary
and environmental exposure in guiding category organization (Abdelrahim
\& Frank, 2024; Colunga \& Smith, 2000; Gershkoff-Stowe \& Smith, 2004;
Jara-Ettinger, Levy, Sakel, Huanca, \& Gibson, 2022; Perry, Samuelson,
Malloy, \& Schiffer, 2010; Larissa K. Samuelson, 2002, 2005; Larissa K.
Samuelson \& Smith, 1999; Yoshida \& Smith, 2003a).

\hypertarget{the-perception-or-conception-debate}{%
\subsection{The Perception or conception
debate}\label{the-perception-or-conception-debate}}

A second key debate focuses on the mechanism or representation
underlying the shape bias. While the tendency to extend nouns based on
shape may be influenced by syntax or statistical regularities, its
precise cognitive basis has been controversial {[}Smith et al. (2002);
Smith, Colunga, \& Yoshida (2010a); Smith, Jones, \& Landau (1996);
Brady \& Chun, 2007; Chun \& Jiang, 1998; Larissa K. Samuelson \& Perone
(2010); Ware \& Booth (2010); A. Booth \& Waxman (2006) ; Susan S. Jones
\& Smith (1993); L. Samuelson \& Horst (2008){]}.

Two competing perspectives attempt to explain this mechanism:

1- Associative and Non-Strategic Mechanism: The shape bias is viewed as
an early cognitive tool, helping children ``break into'' language by
rapidly mapping nouns to referents through associative processes. This
bias operates on a perceptual level, organizing categories around
salient features like shape. For instance, the shape of a dog becomes
synonymous with the category ``dog'' i.e.~a real dog and a plastic toy
dog have the same mental representation. This view posits that the bias
is non-strategic, independent of conceptual understanding or general
world knowledge (Smith et al., 2002; Smith \& Colunga, 2010; Brady \&
Chun, 2007; Chun \& Jiang, 1998).

2- Strategic and Conceptually Controlled Mechanism: The shape bias is
seen as a flexible and controlled process, governed by general world
knowledge and conceptual understanding (A. E. Booth \& Waxman, 2002; A.
Booth, Waxman, \& Huang, 2005). Although a real dog and a plastic toy
dog are both referents of the same label, but they have different mental
representations. In this framework, the bias is a heuristic that can be
overridden when context or conceptual goals require attention to other
features, such as function or material.

\begin{CodeChunk}
\begin{figure}[tb]
\includegraphics[width=1\linewidth]{conceptual_diagram} \caption[A diagram to visualize different factors that potentially contribute to the emergence of the shape bias]{A diagram to visualize different factors that potentially contribute to the emergence of the shape bias}\label{fig:flow_diagram }
\end{figure}
\end{CodeChunk}

\hypertarget{the-nature-of-knowledge}{%
\subsection{The nature of Knowledge}\label{the-nature-of-knowledge}}

Given that task design is influenced by theoretical assumptions, this
raises another important question: What type of knowledge do these tasks
measure? Two key assumptions can stem out of this: Knowledge as Stable
and Fixed: If knowledge is stable, tasks merely elicit pre-existing
constructs. Investigating heterogeneity would then focus on ensuring
task validity and reliability in capturing the theoretical construct.
Knowledge as Dynamic and Task-Dependent: If knowledge is dynamic,
categorization depends on the interaction between task specifics and
children's behavior. This view suggests that children dynamically select
information sources to organize categories, influenced by the context of
the task and the nature of the stimuli (Smith et al., 2010) (Cimpian \&
Markman, 2005; L. Samuelson, 2006; Smith, Colunga, \& Yoshida, 2010b).
If the second assumption holds, achieving consistency in measures across
studies is crucial to isolating the relevant contextual cues that tasks
provide specially for cross-cultural studies that aim to adjudicate
theoretical debates.

Regardless of which assumption holds, it is necessary to evaluate the
heterogeneity in the word categorization studies which highlights the
importance of methodological consistency and the need to consider the
theoretical premises underlying procedural decisions. Understanding how
these premises shape task designs and influence results is key to
advancing our knowledge of label categorization and concept formation.

\hypertarget{current-study}{%
\section{Current Study}\label{current-study}}

This project seeks to evaluate procedural sources of variability as part
of a larger scale assessment of word generalization across age groups
and languages. We utilize a within-subject design, controlling for
individual differences which we believe is important for a proper
comparison between conditions that require giving different instructions
and cues to the participants. We aim to recruit a larger sample size of
a wider age group, with a variety of items and test trials. Given our
focus on early language acquisition and the noun bias dominating early
vocabulary (Frank, Braginsky, Yurovsky, \& Marchman, 2021), we
prioritize studies examining functional information over other types of
conceptual knowledge.

\hypertarget{stimuli-design}{%
\subsection{Stimuli design}\label{stimuli-design}}

To investigate children's reasoning about objects' properties and
functions, a series of object sets were designed, each containing one
exemplar, a material match, a shape match, a function match, and a
distractor (e.g., dax, fep, blicket, gorp, zimbo, wap, blint). These
sets allowed for systematic manipulation of object features to assess
various cognitive processes related to word learning and category
generalization. In Experiment 1, the shape match was contrasted with a
material match. This served as a baseline check and replication of prior
findings regarding shape bias in categorization tasks. In Experiment 2,
the same exemplar was used, but the shape match was contrasted with a
function match.

The function test object was modified in a way that preserved its shape
but altered its functionality (e.g., an object wrapped entirely vs.~one
that could clearly open). Color was excluded across all objects to
ensure that visual similarity was driven solely by shape, material, and
functional cues. Objects were crafted to explore how children reason
about similarity based on whole-object vs.~part-based features (e.g.,
whether specific parts afford a function). Some objects, like the
``Fep,'' ``blint,'' and ``wap,'' were designed with material-critical
functions (e.g., holding water while made of a paper towel). This design
tested whether children could prioritize material when reasoning about
function and to capture the developmental changes. The degree to which
object affordances were visually apparent varied across designs. For
example, The ``Zimbo'' was designed to afford functionality only through
a specific part, while the overall structure was irrelevant. The
``Gorp'' was modeled to resemble objects familiar to slightly older
children, like scissors, allowing exploration of prior experiences'
influence on categorization. This variability was accounted for using
mixed-effects modeling, enabling the examination of how children's
responses were influenced by object features and individual differences.
(An adult similarity rating experiment is currently underway to measure
perceived similarity between objects.)

\hypertarget{experiment-1}{%
\section{Experiment 1:}\label{experiment-1}}

\hypertarget{participants}{%
\subsection{Participants}\label{participants}}

Twenty four typically developing English speaking participants (2-5
years old, mean=44.05, SD=15.29) were recruited from a local nursery
school and children's museum in the US.

\hypertarget{procedure}{%
\subsection{Procedure}\label{procedure}}

Seven trials were conducted in which each participant sees an object
being labeled ``this is a dax'', the object is taken away but still in
view, both test objects and the distractor are displayed simultaneously
while asking the child ``can you find another dax by pointing to it?''.
The child gets to hear the label 3 times while viewing it without
touching it.

\begin{CodeChunk}
\begin{figure}[tb]
\includegraphics[width=1\linewidth]{figs/first_exp-1} \caption[some caption here]{some caption here}\label{fig:first_exp}
\end{figure}
\end{CodeChunk}

\begin{CodeChunk}
\begin{figure}[tb]
\includegraphics[width=1\linewidth]{figs/first_exp_stim-1} \caption[some caption here]{some caption here}\label{fig:first_exp_stim}
\end{figure}
\end{CodeChunk}

\hypertarget{results}{%
\subsection{Results}\label{results}}

Participants showed an overall shape bias across all trials (shape:61\%,
material:30\%, distractor: 9\%). Figure \ref{fig:first_exp} shows a
developmental shift to choose by shape by age 3, replicating what is
seen previously in the literature.\\
A generalized logistic mixed-effects model (GLMM) reveals an average
intercept odds of 0.11 (odds of 0.11:1 at the mean age, \(p=\)
\textless{} .001), with a significant increase in odds of 1.06 per unit
increase in age (\(p=\) \textless{} .001). The model also shows
variability at the item-level intercept (variance = 0.11, SD = 0.32)
across 7 unique items (standardlabel groups). After replicating the
shape bias effect using the set of stimuli we created in a simple set
up, our next experiment explores an design that tests for both
conditions when shape is only contrasted with material without any
additional information, and a condition in which shape is contrasted
with function after demonstrating the function for the exemplar, while
controlling for individual differences with a bigger sample size to
capture variability at the item level.

\hypertarget{discussion}{%
\subsection{Discussion:}\label{discussion}}

\hypertarget{experiment-2}{%
\section{Experiment 2}\label{experiment-2}}

\hypertarget{participants-1}{%
\subsection{Participants}\label{participants-1}}

31 (target n=96, 24 per each age group) participants between 2-5 years
old (mean=48.22, SD=5.54, n per age group) were recruited from a local
nursery school in the US.

\hypertarget{procedure-1}{%
\subsection{Procedure}\label{procedure-1}}

A within subject manipulation with two conditions: material or function.
The material condition is identical to the first experiment. In the
function condition, the experimenter introduce the exemplar object
``this is a dax'', gives the child 15 seconds seconds to play with it,
provides functional information `` the dax grapes toys'', gives another
15 seconds to play with it, and puts the toy away but within view,
before introducing the test objects and asks for a response.

\begin{CodeChunk}
\begin{figure*}[!h]
\includegraphics[width=1\linewidth]{figs/jitter_function-1} \caption[Experiment 2, function vs]{Experiment 2, function vs. no function 'material'  condition. Children choose by shape more, even when function information is made salient}\label{fig:jitter_function}
\end{figure*}
\end{CodeChunk}

\begin{CodeChunk}
\begin{figure}[tb]
\includegraphics[width=1\linewidth]{figs/sec_exp_stim-1} \caption[Experiment 2, proportion of choosing by each dimension per exemplar item 'indicated by its novel label]{Experiment 2, proportion of choosing by each dimension per exemplar item 'indicated by its novel label. We note variability across items}\label{fig:sec_exp_stim}
\end{figure}
\end{CodeChunk}

\hypertarget{preliminary-results}{%
\subsection{Preliminary results}\label{preliminary-results}}

Similar to what is conveyed in Figure \ref{fig:jitter_function}, a
generalized logistic mixed-effects model (GLMM) showed a lower baseline
odds of the shape bias in the material condition compared to the
function, and the odds ratio increases with age. In additon, random
effects indicate variability in intercepts across participants (SD =
0.22) and across items (SD = 1.38) for 31 participants and 7 items as in
Figure \ref{fig:sec_exp_stim}. (Notably, the confidence intervals show
uncertainty ``include 0'', however data collection is still ongoing.)

\hypertarget{discussion-1}{%
\section{Discussion}\label{discussion-1}}

The word extension and category organization literature is highly
heterogeneous. Studies in this domain lack an integrative and
commensurable design, which hinders our ability to draw consistent
conclusions. To achieve an accurate measurement of category organization
and concept learning, we need a reliable and valid wide range set of
stimuli objects, consistent task formats and test designs, can be used
in multi-site cross-cultural experiments unified across laboratories to
maximally account for the variability. Our evaluation of the word
extension literature reveals that making procedural decisions, which we
think are likely a primary source of unexplained variability, is
unattainable without running a series of controlled experiments that
would allow us to systematically assess how different designs and
stimuli covary with response patterns. As discussed above, as well as in
many existing studies, the objects used in word extension studies vary
on many dimensions, relatd to their ontological status ``a solid or
non-solid'', their animacy, their complexity, and their affordances. In
the case of studies that highlights the relevance of function
information or general affordances, objects varied on whether the
properties of the objects were structurally independent i.e.~whether the
function was afforded by a part of the object or the texture of the
object, or whether they are designed to be construed as artifact-like
objects with properties not immediately available upon visual
inspection, or whether affordances were intrinsic to the objects (cite
Graham and diesendruck 2010). In our initial results, we see a strong
tendency to generalize by shape, even in conditions designed to make
function salient. This suggests that, even with a potential saliency
effect, where the trials highlighted functional information, it failed
to override the preference for shape-based choices. In additon, many
children explored whether their chosen test object could perform the
intended function after selecting it based on shape, specially in case
of designs by which the visual inspection of the affordance was not
certain (the dax for example. This behavior implies that the shape-based
selection might not reflect a disregard for functional information but
rather a hypothesis that objects sharing shape might also share
functionality. In the case of the ``Zimbo'', the object was designed in
away that makes the function afforded only by a part, but the whole
structure is irrelavant. Hence, we see children go with function in the
function condition, however, the objects shared the shape of a small
part that is critical for performing the function. This complicates the
interpretation of the results as the shape of the part but not the whole
is shared across chosen objects. For evaulating the children's ability
to reason about intrinsic affordances, we created the ``Fep'' in a way
such that the material itself is very critical for performing the
function ``holding water while being made of paper towel''. We are EDIT:
What does it mean to talk about validity and reliability in a task like
the word extension one? What is the construct we are talking about here.
It is a measure at the group level rather than at the individual level
\textgreater\textgreater{} this kid had this level of shape bias.
\textgreater\textgreater{} is it an issue of the signal is not even
there, or is it a matter of the level/degree of the signal when it comes
to cross-cultural work. (Madole, K. L., \& Oakes, L. M. (1999)):
Although the distinction between perceptual and conceptual categories
makes a certain intuitive sense, it may only confuse our attempts to
understand psychological reality. Establishing a reliable metric for
this distinction is extremely difficult and attempts to operationalize
the terms perceptual and conceptual are invariably highly ambiguous and
task-specific. How the perceptual vs conceptual debate maps into the
cross cultural differences debate ? As we mentioned in the introduction,
procedural variation observed in the literature have followed
theoretical debates and in fact reinforced by them. Thus, evaluating the
theories given the state of evidence is not feasbile and it will endup
crashing with this heterogeneity and with the broader question on the
type of knowledge being highlighted by each one of the evidence.

\hypertarget{references}{%
\section{References}\label{references}}

\setlength{\parindent}{-0.1in} 
\setlength{\leftskip}{0.125in}

\noindent

\hypertarget{refs}{}
\begin{CSLReferences}{1}{0}
\leavevmode\vadjust pre{\hypertarget{ref-abdelrahim_frank_2024}{}}%
Abdelrahim, S., \& Frank, M. C. (2024, September). Examining the
robustness and generalizability of the shape bias: A meta-analysis.
PsyArXiv.
http://doi.org/\href{https://doi.org/10.31234/osf.io/3by54}{10.31234/osf.io/3by54}

\leavevmode\vadjust pre{\hypertarget{ref-Baldwin1992ClarifyingTR}{}}%
Baldwin, D. A. (1992). Clarifying the role of shape in children's
taxonomic assumption. \emph{Journal of Experimental Child Psychology},
\emph{54 3}, 392--416. Retrieved from
\url{https://api.semanticscholar.org/CorpusID:29725812}

\leavevmode\vadjust pre{\hypertarget{ref-BOOTH2002B11}{}}%
Booth, A. E., \& Waxman, S. R. (2002). Word learning is {``smart''}:
Evidence that conceptual information affects preschoolers' extension of
novel words. \emph{Cognition}, \emph{84}(1), B11--B22.
http://doi.org/\url{https://doi.org/10.1016/S0010-0277(02)00015-X}

\leavevmode\vadjust pre{\hypertarget{ref-dejavu}{}}%
Booth, A., \& Waxman, S. (2006). Déjà vu all over again: Re-revisiting
the conceptual status of early word learning: Comment on smith and
samuelson (2006). \emph{Developmental Psychology}, \emph{42}, 1344--6.
http://doi.org/\href{https://doi.org/10.1037/0012-1649.42.6.1344}{10.1037/0012-1649.42.6.1344}

\leavevmode\vadjust pre{\hypertarget{ref-2005_Booth}{}}%
Booth, A., Waxman, S., \& Huang, Y. (2005). Conceptual information
permeates word learning in infancy. \emph{Developmental Psychology},
\emph{41}(3), 491--505.
http://doi.org/\href{https://doi.org/10.1037/0012-1649.41.3.491}{10.1037/0012-1649.41.3.491}

\leavevmode\vadjust pre{\hypertarget{ref-Centner2003OnRM}{}}%
Centner, D. (2003). On relational meaning : The acquisition of verb
meaning. In. Retrieved from
\url{https://api.semanticscholar.org/CorpusID:7539638}

\leavevmode\vadjust pre{\hypertarget{ref-cimpian2005absence}{}}%
Cimpian, A., \& Markman, E. M. (2005). The absence of a shape bias in
children's word learning. \emph{Developmental Psychology}, \emph{41}(6),
1003.

\leavevmode\vadjust pre{\hypertarget{ref-CLARK197365}{}}%
CLARK, E. V. (1973). WHAT's IN a WORD? ON THE CHILD's ACQUISITION OF
SEMANTICS IN HIS FIRST LANGUAGE11This research was supported in part by
NSF grant GS-1880 to the language universals project, stanford
university, and in part by NSF grant GS-30040 to the author. In T. E.
Moore (Ed.), \emph{Cognitive development and acquisition of language}
(pp. 65--110). San Diego: Academic Press.
http://doi.org/\url{https://doi.org/10.1016/B978-0-12-505850-6.50009-8}

\leavevmode\vadjust pre{\hypertarget{ref-colunga2000learning}{}}%
Colunga, E., \& Smith, L. B. (2000). Learning to learn words: A
cross-linguistic study of the shape and material biases. In
\emph{Proceedings of the 24th annual boston university conference on
language development} (Vol. 1, pp. 197--207).

\leavevmode\vadjust pre{\hypertarget{ref-frank2021}{}}%
Frank, M. C., Braginsky, M., Yurovsky, D., \& Marchman, V. A. (2021).
\emph{Variability and consistency in early language learning: The
wordbank project}. The MIT Press.
http://doi.org/\href{https://doi.org/10.7551/mitpress/11577.001.0001}{10.7551/mitpress/11577.001.0001}

\leavevmode\vadjust pre{\hypertarget{ref-gathercole_1997}{}}%
Gathercole, V. C. M., \& Min, H. (1997). Word meaning biases or
language-specific effects? {Evidence} from {English}, {Spanish} and
{Korean}. \emph{First Language}, \emph{17}(51), 031--56.
http://doi.org/\href{https://doi.org/10.1177/014272379701705102}{10.1177/014272379701705102}

\leavevmode\vadjust pre{\hypertarget{ref-gershkoff2004shape}{}}%
Gershkoff-Stowe, L., \& Smith, L. B. (2004). Shape and the first hundred
nouns. \emph{Child Development}, \emph{75}(4), 1098--1114.

\leavevmode\vadjust pre{\hypertarget{ref-graham_2010}{}}%
Graham, S. A., \& Diesendruck, G. (2010). Fifteen-month-old infants
attend to shape over other perceptual properties in an induction task.
\emph{Cognitive Development}, \emph{25}(2), 111--123.
http://doi.org/\href{https://doi.org/10.1016/j.cogdev.2009.06.002}{10.1016/j.cogdev.2009.06.002}

\leavevmode\vadjust pre{\hypertarget{ref-Graham1999InfantsRO}{}}%
Graham, S. A., \& Poulin-Dubois, D. (1999). Infants' reliance on shape
to generalize novel labels to animate and inanimate objects.
\emph{Journal of Child Language}, \emph{26}, 295--320. Retrieved from
\url{https://api.semanticscholar.org/CorpusID:43424185}

\leavevmode\vadjust pre{\hypertarget{ref-GRAHAM1999128}{}}%
Graham, S. A., Williams, L. D., \& Huber, J. F. (1999). Preschoolers'
and adults' reliance on object shape and object function for lexical
extension. \emph{Journal of Experimental Child Psychology},
\emph{74}(2), 128--151.
http://doi.org/\url{https://doi.org/10.1006/jecp.1999.2514}

\leavevmode\vadjust pre{\hypertarget{ref-imai1997}{}}%
Imai, M., \& Gentner, D. (1997). A cross-linguistic study of early word
meaning: Universal ontology and linguistic influence. \emph{Cognition},
\emph{62}(2), 169--200.

\leavevmode\vadjust pre{\hypertarget{ref-imai_childrens_1994}{}}%
Imai, M., Gentner, D., \& Uchida, N. (1994). Children's theories of word
meaning: {The} role of shape similarity in early acquisition.
\emph{Cognitive Development}, \emph{9}(1), 45--75.
http://doi.org/\href{https://doi.org/10.1016/0885-2014(94)90019-1}{10.1016/0885-2014(94)90019-1}

\leavevmode\vadjust pre{\hypertarget{ref-jara2022}{}}%
Jara-Ettinger, J., Levy, R., Sakel, J., Huanca, T., \& Gibson, E.
(2022). The origins of the shape bias: Evidence from the tsimane'.
\emph{Journal of Experimental Psychology: General}.

\leavevmode\vadjust pre{\hypertarget{ref-Jones2003}{}}%
Jones, Susan S. (2003). Late talkers show no shape bias in a novel name
extension task. \emph{Developmental Science}, \emph{6}(5), 477--483.
http://doi.org/\url{https://doi.org/10.1111/1467-7687.00304}

\leavevmode\vadjust pre{\hypertarget{ref-JONES1993113}{}}%
Jones, Susan S., \& Smith, L. B. (1993). The place of perception in
children's concepts. \emph{Cognitive Development}, \emph{8}(2),
113--139.
http://doi.org/\url{https://doi.org/10.1016/0885-2014(93)90008-S}

\leavevmode\vadjust pre{\hypertarget{ref-JONES1998323}{}}%
Jones, Susan S., \& Smith, L. B. (1998). How children name objects with
shoes. \emph{Cognitive Development}, \emph{13}(3), 323--334.
http://doi.org/\url{https://doi.org/10.1016/S0885-2014(98)90014-4}

\leavevmode\vadjust pre{\hypertarget{ref-JONES_SMITH_2005}{}}%
JONES, S. S., \& SMITH, L. B. (2005). Object name learning and object
perception: A deficit in late talkers. \emph{Journal of Child Language},
\emph{32}(1), 223--240.
http://doi.org/\href{https://doi.org/10.1017/S0305000904006646}{10.1017/S0305000904006646}

\leavevmode\vadjust pre{\hypertarget{ref-Kucker2019ReproducibilityAA}{}}%
Kucker, S. C., Samuelson, L. K., Perry, L. K., Yoshida, H., Colunga, E.,
Lorenz, M. G., \& Smith, L. B. (2019). Reproducibility and a unifying
explanation: Lessons from the shape bias. \emph{Infant Behavior \&
Development}, \emph{54}, 156--165. Retrieved from
\url{https://api.semanticscholar.org/CorpusID:53045726}

\leavevmode\vadjust pre{\hypertarget{ref-landau1996}{}}%
Landau, B., \& Jones, S. (1998). Object shape, object function, and
object name. \emph{Journal of Memory and Language - J MEM LANG},
\emph{38}, 1--27.
http://doi.org/\href{https://doi.org/10.1006/jmla.1997.2533}{10.1006/jmla.1997.2533}

\leavevmode\vadjust pre{\hypertarget{ref-LANDAU1988299}{}}%
Landau, B., Smith, L. B., \& Jones, S. S. (1988). The importance of
shape in early lexical learning. \emph{Cognitive Development},
\emph{3}(3), 299--321.
http://doi.org/\url{https://doi.org/10.1016/0885-2014(88)90014-7}

\leavevmode\vadjust pre{\hypertarget{ref-Merriman_Scott_Marazita_1993}{}}%
Merriman, W. E., Scott, P. D., \& Marazita, J. (1993). An
appearance-function shift in children's object naming. \emph{Journal of
Child Language}, \emph{20}(1), 101--118.
http://doi.org/\href{https://doi.org/10.1017/S0305000900009144}{10.1017/S0305000900009144}

\leavevmode\vadjust pre{\hypertarget{ref-Nelson1974ConceptWA}{}}%
Nelson, K. (1974). Concept, word, and sentence: Interrelations in
acquisition and development. \emph{Psychological Review}, \emph{81},
267--285. Retrieved from
\url{https://api.semanticscholar.org/CorpusID:143965074}

\leavevmode\vadjust pre{\hypertarget{ref-perry2010learn}{}}%
Perry, L. K., Samuelson, L. K., Malloy, L. M., \& Schiffer, R. N.
(2010). Learn locally, think globally: Exemplar variability supports
higher-order generalization and word learning. \emph{Psychological
Science}, \emph{21}(12), 1894--1902.

\leavevmode\vadjust pre{\hypertarget{ref-Rescorla1980OverextensionIE}{}}%
Rescorla, L. (1980). Overextension in early language development.
\emph{Journal of Child Language}, \emph{7}, 321--335. Retrieved from
\url{https://api.semanticscholar.org/CorpusID:17145854}

\leavevmode\vadjust pre{\hypertarget{ref-replycimpian}{}}%
Samuelson, L. (2006). An attentional learning account of the shape bias:
Reply to cimpian and markman (2005) and booth, waxman, and huang (2005).
\emph{Developmental Psychology}, \emph{42}, 1339--43.
http://doi.org/\href{https://doi.org/10.1037/0012-1649.42.6.1339}{10.1037/0012-1649.42.6.1339}

\leavevmode\vadjust pre{\hypertarget{ref-samuelson_statistical_2002}{}}%
Samuelson, Larissa K. (2002). Statistical regularities in vocabulary
guide language acquisition in connectionist models and 15-20-month-olds.
\emph{Developmental Psychology}, \emph{38}(6), 1016--1037.
http://doi.org/\href{https://doi.org/10.1037/0012-1649.38.6.1016}{10.1037/0012-1649.38.6.1016}

\leavevmode\vadjust pre{\hypertarget{ref-2005_Samuelson}{}}%
Samuelson, Larissa K. (2005). Statistical regularities in vocabulary
guide language acquisition in connectionist models and 15-20-month-olds.
\emph{Developmental Psychology}, \emph{38}(6), 1016--1037.
http://doi.org/\href{https://doi.org/10.1037/0012-1649.38.6.1016}{10.1037/0012-1649.38.6.1016}

\leavevmode\vadjust pre{\hypertarget{ref-samuelson2008rigid}{}}%
Samuelson, Larissa K., Horst, J. S., Schutte, A. R., \& Dobbertin, B. N.
(2008). Rigid thinking about deformables: Do children sometimes
overgeneralize the shape bias? \emph{Journal of Child Language},
\emph{35}(3), 559--589.

\leavevmode\vadjust pre{\hypertarget{ref-SAMUELSON2010138}{}}%
Samuelson, Larissa K., \& Perone, S. (2010). Rethinking
conceptually-based inference --- grounding representation in task and
behavioral dynamics: Commentary on {``fifteen-month-old infants attend
to shape over other perceptual properties in an induction task,''} by s.
Graham and g. Diesendruck, and {``form follows function: Learning about
function helps children learn about shape,''} by e. Ware and a. booth.
\emph{Cognitive Development}, \emph{25}(2), 138--148.
http://doi.org/\url{https://doi.org/10.1016/j.cogdev.2010.02.002}

\leavevmode\vadjust pre{\hypertarget{ref-samuelson1999}{}}%
Samuelson, Larissa K., \& Smith, L. B. (1999). Early noun vocabularies:
Do ontology, category structure and syntax correspond? \emph{Cognition},
\emph{73}(1), 1--33.

\leavevmode\vadjust pre{\hypertarget{ref-samuelson2000children}{}}%
Samuelson, Larissa K., \& Smith, L. B. (2000). Children's attention to
rigid and deformable shape in naming and non-naming tasks. \emph{Child
Development}, \emph{71}(6), 1555--1570.

\leavevmode\vadjust pre{\hypertarget{ref-dynamicBias}{}}%
Samuelson, L., \& Horst, J. (2008). Shape bias special section:
Confronting complexity: Insights from the details of behavior over
multiple timescales. \emph{Developmental Science}, \emph{11}, 209--15.
http://doi.org/\href{https://doi.org/10.1111/j.1467-7687.2007.00667.x}{10.1111/j.1467-7687.2007.00667.x}

\leavevmode\vadjust pre{\hypertarget{ref-Smith2003MakingAO}{}}%
Smith, L. B., Colunga, E., \& Yoshida, H. (2003). Making an ontology:
Cross-linguistic evidence. In. Retrieved from
\url{https://api.semanticscholar.org/CorpusID:195941919}

\leavevmode\vadjust pre{\hypertarget{ref-smithcolunga2010}{}}%
Smith, L. B., Colunga, E., \& Yoshida, H. (2010a). Knowledge as process:
Contextually cued attention and early word learning. \emph{Cognitive
Science}, \emph{34}(7), 1287--1314.
http://doi.org/\url{https://doi.org/10.1111/j.1551-6709.2010.01130.x}

\leavevmode\vadjust pre{\hypertarget{ref-smithContext}{}}%
Smith, L. B., Colunga, E., \& Yoshida, H. (2010b). Knowledge as process:
Contextually cued attention and early word learning. \emph{Cognitive
Science}, \emph{34}(7), 1287--1314.
http://doi.org/\url{https://doi.org/10.1111/j.1551-6709.2010.01130.x}

\leavevmode\vadjust pre{\hypertarget{ref-Smith1996NamingIY}{}}%
Smith, L. B., Jones, S. S., \& Landau, B. (1996). Naming in young
children: A dumb attentional mechanism? \emph{Cognition}, \emph{60},
143--171. Retrieved from
\url{https://api.semanticscholar.org/CorpusID:18659784}

\leavevmode\vadjust pre{\hypertarget{ref-smith_object_2002}{}}%
Smith, L. B., Jones, S. S., Landau, B., Gershkoff-Stowe, L., \&
Samuelson, L. (2002). Object name learning provides on-the-job training
for attention. \emph{Psychological Science}, \emph{13}(1), 13--19.
http://doi.org/\href{https://doi.org/10.1111/1467-9280.00403}{10.1111/1467-9280.00403}

\leavevmode\vadjust pre{\hypertarget{ref-soja1991ontological}{}}%
Soja, Nancy N., Carey, S., \& Spelke, E. S. (1991). Ontological
categories guide young children's inductions of word meaning: Object
terms and substance terms. \emph{Cognition}, \emph{38}(2), 179--211.

\leavevmode\vadjust pre{\hypertarget{ref-soja_perception_1992}{}}%
Soja, Nancy N., Carey, S., \& Spelke, E. S. (1992). Perception,
ontology, and word meaning. \emph{Cognition}, \emph{45}(1), 101--107.
http://doi.org/\href{https://doi.org/10.1016/0010-0277(92)90025-D}{10.1016/0010-0277(92)90025-D}

\leavevmode\vadjust pre{\hypertarget{ref-subrahmanyam_2006}{}}%
Subrahmanyam, K., \& Chen, H.-H. N. (2006). A crosslinguistic study of
children's noun learning: {The} case of object and substance words.
\emph{First Language}, \emph{26}(2), 141--160.
http://doi.org/\href{https://doi.org/10.1177/0142723706060744}{10.1177/0142723706060744}

\leavevmode\vadjust pre{\hypertarget{ref-TekAutism}{}}%
Tek, S., Jaffery, R., Fein, D., \& Naigles, L. (2008). Do children with
autism show a shape bias in word learning? \emph{Autism Research :
Official Journal of the International Society for Autism Research},
\emph{1}, 208--22.
http://doi.org/\href{https://doi.org/10.1002/aur.38}{10.1002/aur.38}

\leavevmode\vadjust pre{\hypertarget{ref-TekAutismLessons}{}}%
Tek, S., \& Naigles, L. (2017). The shape bias as a word-learning
principle: Lessons from and for autism spectrum disorder.
\emph{Translational Issues in Psychological Science}, \emph{3}, 94--103.
http://doi.org/\href{https://doi.org/10.1037/tps0000104}{10.1037/tps0000104}

\leavevmode\vadjust pre{\hypertarget{ref-WARE2010124}{}}%
Ware, E. A., \& Booth, A. E. (2010). Form follows function: Learning
about function helps children learn about shape. \emph{Cognitive
Development}, \emph{25}(2), 124--137.
http://doi.org/\url{https://doi.org/10.1016/j.cogdev.2009.10.003}

\leavevmode\vadjust pre{\hypertarget{ref-yoshida2003known}{}}%
Yoshida, H., \& Smith, L. B. (2003a). Known and novel noun extensions:
Attention at two levels of abstraction. \emph{Child Development},
\emph{74}(2), 564--577.

\leavevmode\vadjust pre{\hypertarget{ref-yoshida2003}{}}%
Yoshida, H., \& Smith, L. B. (2003b). Shifting ontological boundaries:
How japanese- and english-speaking children generalize names for animals
and artifacts. \emph{Developmental Science}, \emph{6}(1), 1--17.
http://doi.org/\url{https://doi.org/10.1111/1467-7687.00247/_1}

\end{CSLReferences}

\bibliographystyle{apacite}


\end{document}
